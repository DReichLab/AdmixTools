\documentclass{article} %% see ~/biology/admix/mollydir and orchestra
\newcommand  {\cH}  {{\cal H}}
\newcommand  {\cS}  {{\cal S}}
\newcommand  {\cB}  {{\cal B}}
\newcommand  {\cL}  {{\cal L}}
\newcommand  {\cF}  {{\cal F}}
\newcommand  {\bpsi}  {{{\bf \psi}}}
\newcommand  {\bpm}  {{{\bpsi_{-1, -2}}}}
\newcommand  {\bphi}  {{{\bf \phi}}}
\newcommand  {\bpi}  {{{\bf \pi}}}
\newcommand  {\bzero}  {{{\bf 0}}}
\newcommand  {\np}  {{\ \newpage}}
\newcommand  {\psmall}  {{$< {10}^{-6}$}}
\newcommand  {\psmallx}  {{< {10}^{-6}}}
\newcommand  {\Nk}  {{N^{[k]}}}
\newcommand  {\hNk}  {{{\hat N}^{[k]}}}
\newcommand  {\Dk}  {{D^{[k]}}}
\newcommand  {\hDk}  {{{\hat D}^{[k]}}}
\newcommand  {\kstar}    {{{K ^ \star}}} 
\newcommand  {\lstar}    {{{L ^ \star}}} 
\newcommand  {\phistar}    {{{\bphi ^ \star}}} 
\newcommand  {\kk}    {{{[k]}}}
\newcommand  {\hathw}  {{{\hat h}_W}}
\newcommand  {\hathx}  {{{\hat h}_X}}
\newcommand  {\half}  {{\frac{1}{2}}}
\newcommand  {\nl}  {{\ \newline}}
\newcommand  {\nld }  {{\ \newline \noindent}}
\newcommand  {\qpw }  {{\em qpWave }} 
\newcommand  {\qpa }  {{\em qpAdm }} 

\usepackage[dvips]{graphicx}

\begin{document}
\bibliographystyle{plain}

\title{Documentation for {\em qpWave} and {\em qpAdm}} 
\author{Nick Patterson}
\maketitle
\section{Introduction} 
We document 2 programs: {\em qpWave} and {\em qpAdm} based on 
a common set of ideas related to $f_4$ statistics \cite{ancadm} 
The first program, \qpw (formerly {\em qp4wave2}) emerged from work with 
David Reich on the peopling of the Americas \cite{natam1}.
The second, \qpa is more recent, and is an attempt to systematize ideas 
of Iosif Lazaridis, using $f_4$ statistics in a regression context, but 
also incorporating methods from \qpw.

\nld
In \cite[S6]{natam1} we showed that if we took a set of $a$ {\em left} populations $U$ 
and a set of $b$ {\em right}  populations $V$ and considered the matrix 
\[
X(u, v) = F_4(u_0, u; v_0, v) 
\]
where $u_0, v_0$ are some fixed populations of $U$ and $V$, and 
$l, r$ range over all choices of populations of $U$, $V$.  
We can assume that $u \ne u_0, v \ne v_0$, so that the 
matrix $X$ is $ (a-1) \times (b-1) $.
We then showed that if $X$ had rank $r$ and there had been $n$ waves of immigration from $V$ to  $U$ 
with no back-migration from $U$ to $V$, 
then:     
\[ 
r + 1 \le n
\]
In our initial application we used this to show that there must have been at least 3 waves 
of immigration into the (pre-Columbus) Americas.  

\nld
\section{Algorithmic details for \qpw}
We describe our computational strategy in a little more detail.  
We compute $\hat X$,  an estimate of $X$ so that in the notation of \cite{ancadm} 
\[
\hat X(u, v) = f_4(u_0, u; v_0, v) 
\]
We can use the block jackknife \cite{blockjack1}
to compute $V$ an estimate of the error covariance of $X$.
To test if $\hat X$ has rank $r$ we write 
\[
\hat X = A . B  + E 
\]
where $A$ is $(a-1) \times r $, $B$ is $(e \times (b-1)$ and $E$ is a matrix of residuals.  
The (log) likelihood for ($A, B$) and implicitly $r$ is: 
\[
\cL(A, B) = -\half \sum_{i,j,k,l} 
V_{i,j; k,l}^{-1} 
E_{i,j} E_{k,l} 
\]
where the residual matrix $E$ is defined by 
\[
E = \hat X - A.B 
\]
For each $r$ we set $A, B$ initially by an SVD analysis of $X$, 
and then iterate, minimizing $\cL$ with respect to $A$, $B$ in turn. 
For fixed $A$, $\cL(A,B)$ is quadratic in $B$ and can be minimized by 
solving linear equations.
Since $A, B$ only enter into the likelihood though a matrix product, we 
see that 
\[
A.B = (A.Q) . (Q^{-1} B 
\]
for any non-singular $r \times r$ matrix $Q$.  
Thus the number of degrees of freedom is 
\[
d((r) = ((a-1) + (b-1) )r - r * r = r (a+b-(r+2)) 
\]
As a check, if $r$ is the maximal rank $Min(a-1, b-1)$, then 
$d(r) = (a-1)(b-1)$ which is obviously correct.  This is 
the {\em saturated} model, where we fit the data perfectly. 

\nld
We compute statistics with a likelihood ratio test.   
\section{Parameters and output of \qpw} 
Here is a sample parameter file.  
\begin{verbatim} 
DIR:     /home/np29/broaddata/bl14
S1:              honjp    
indivname:       DIR/S1.ind  
snpname:         DIR/S1.snp       
genotypename:    DIR/S1.geno   
badsnpname:      ./cpgmf
popleft:  pleft   
popright: pright 
maxrank: 4 
## not needed here
\end{verbatim} 
The top lines are parameters that will likely be familiar, 
for example they are the same in {em convertf, qpDstat, qp3Pop}.  
In this run I did not want to use CpG sites, which are removed by the badsnpname: line. 
pleft is a file of populations 1/line, pright also.  We have 
\nld
pleft: 
\begin{verbatim} 
WHG
LBKNeolithic
YamnayaEBA
\end{verbatim}
while the right population list was: 
\nld
pright:
\begin{verbatim} 
Han
Eskimo
Mbuti
Karitiana
Kharia
Onge
Ulchi
\end{verbatim}
(the right set of populations are chosen so that they are differently related to 
West Eurasia).  
Extracts from the output:
\begin{verbatim}
f4rank: 0 dof: 12 chisq: 330.440 tail:  1.86337038e-63 
 dofdiff:  0 chisqdiff:  0.000 taildiff:                1
f4rank: 1 dof:  5 chisq:  46.979 tail:  5.73674279e-09 
 dofdiff:  7 chisqdiff: 83.460 taildiff:   2.05163995e-57 
f4rank: 2 dof:  0 chisq:   0.000 tail:               1 
 dofdiff:  5 chisqdiff: 46.979 taildiff:   5.73674279e-09 
\end{verbatim}
For each line we rank a $\chi^2$ statistic and tail area (chisq and tail) comparing with the saturated model, and also 
a chi-square statistic and tail for the model with one rank less.  
We see here that the rank 1 model has a $p-value$ of $5.7 \times {10}^{-9}$, comparing with the saturated model  and can be rejected. 
We have very strong evidence here that {\em WHG, LBKNeolithic, YamnayaEBA} are not the product of 2 waves from outside West Eurasia. o

The matrices $A, B$ are published and may be useful.
\begin{verbatim}
B:
          scale     1.000     1.000 
         Eskimo     1.323    -0.011 
          Mbuti    -0.306     2.300 
      Karitiana     1.995     0.262 
         Kharia     0.190     0.592 
           Onge    -0.206    -0.532 
          Ulchi     0.308    -0.082 
A:
          scale  1392.604  1651.101 
   LBKNeolithic    -0.533     1.310 
     YamnayaEBA     1.310     0.533 
\end{verbatim}
We show here $A, B$ matrices for the saturated model.  (Actually we show $transpos(B)$, with a scale factor for the columns. 
From the second column  Mbuti has a large coefficient, and $LBKNeolithic$ also.  
From the first column we see Karitiana and YamnayaEBA.  
It is therefore not surprising to see from {\em qpDstat} output

\begin{verbatim}
                                      D        Z
WHG  LBKNeolithic  Han      Mbuti   0.0208   7.780  
WHG    YamnayaEBA  Han  Karitiana   0.0239   7.627 
WHG    YamnayaEBA  Han      Mbuti   0.0053   1.765 
\end{verbatim}
with the first 2 $Z$ scores large, the last much smaller.

\nld
We note that the $\chi^2$ statistics here, using the LRT are computed using a fixed covariance $V$.  It would be formally 
more correct to reestimate $V$,  simultaneously with $A, B$.  This would greatly increase complexity, without 
adding much precision.  

\nld
{
\em I strongly recommend attempting to keep the population lists here small.  If $a, b$ are large then the covariance $V$ is 
a big matrix, and in practice will be estimated poorly.  This can be expected to lead to trouble. 
}

\section{Finding mixture coefficients --- \qpa}
\nld
We next describe a novel idea for finding admixture weights using $f_4$ statistics.  
This was motivated by work of Iosif Lazaridis, though the details here are quite distinct. 
Let $T$ be a {\em target} population, $S = \{s_1, s_2, \ldots s_n \}$ a set of source populations.  
In the easiest case to consider, when $T$ is an an admixture of populations of $S$ we can 
write symbolically 
\[
T = \sum_{i=1}^n w_i s_i 
\] 
It then follows that for any populations $r_1, r_2$ 
\begin{eqnarray*} 
\sum_i w_i f_4(T, s_i, r_1, r_2) & = & f_4(T, T, r_1, r_2)  \\
& = & 0 
\end{eqnarray*}
A little thought shows that this is true even if 
the populations $s_i$ are descendents of the true source populations, 
provided that there has been no gen flow between the most recent ancestor of $T, S$ on the one hand 
and ancestor of $r_1, r_2$ on the other.  
[In passing, we note that we used $f_3$ statistics in \cite{capec1} to derive mixing coefficients
for modern admixture.  The methods there require samples of the actual source and admixed populations, 
but do not require outgroup populations as we do here with the $r_i$.]

\nld
Thus, if $T$ is admixed, as above, pick a set of outgroup populations $R$, and 
\begin{enumerate} 
\item
Check, using \qpw, setting left populations $L=S$, and right populations $R$ that 
the matrix $X$ has full rank $n-1$.  
\item
Check, again using \qpw, that letting $L = \{T, S\}$ the re is no strong evidence that 
the rank of $X$ increases.
\end{enumerate}
We now will take $T$ as the base population of $L = \{T, S\}$, which simplifies the algebra.  
We calculate matrices $A, B$ as in \qpw, with the rank set to $n-2$ (corank $1$).   
So the recovered $A$ is of dimensions $ (n-1) \times (n-2)$.    
It then follows that 
estimates $\bf w = (w_1, w_2, \ldots w_n)$ of the admixture weights can be found by solving 
the equations: 
\begin{eqnarray*}
\bf w. A & =  & \bf 0  \\
\sum_{i=1}^n w_i = 1 
\end{eqnarray*}
We can use the block jackknife to compute a covariance matrix for the errors.  
[Formally, we should reestimate $V$ as we delete blocks in the jackknife.  
This is not
presently done, as it would add complexity and seems unlikely to make a material difference.]
\subsection{All subsets regression}
Suppose $U$ is a proper) subset of $S$.  It is interesting to require that $w_i = 0$ if $s_i \in U$.  
That is populations of $U$ do not contribute to the admixture of $T$.  This constrains the 
structure of the matrix $A$ but optimization is still easy to carry through.   
It can be shown that if $|U| = f$, then the saturated model has $ (b-a) + f + 1$  degrees of freedom.  
Since in practice $n$ will be small, it is computationally reasonable to 
try all proper  subsets of $S$;  for each we can compute the best coefficients and a chi-square score
using an LRT.

\
Here is a sample parameter file.  
\begin{verbatim} 
DIR:     /home/np29/broaddata/bl14
S1:              honjp    
indivname:       DIR/S1.ind  
snpname:         DIR/S1.snp       
genotypename:    DIR/S1.geno   
badsnpname:      ./cpgmf
popleft:  pleftx   
popright: pright 
maxrank: 4 
## not needed here
\end{verbatim} 
The format of the parameter file is identical to that for \qpw. 
qpop1 is a file of populations 1/line, superpops also.  We have 
\nld
pleftx: 
\begin{verbatim} 
CordedWareNeolithic
WHG
LBKNeolithic
YamnayaEBA
\end{verbatim}
while the right population list was the same as {\em pright} 
described in the section on parameters of \qpw. 

\nld
BY convention the first population of the left list is the target. 
So here we are examining {\em CordedWareNeolithic} as a mixture of the 
other three populations.

\nld
Extracts from the output:
We begin by testing using \qpw methodology whether a rank 2 matrix can be accepted.  Here we get a p-value of $0.07$ 
and proceed. 
\begin{verbatim}
f4rank: 2 dof: 4 chisq: 8.647 tail: 0.0705644793 
 dofdiff: 6 chisqdiff: -8.647 taildiff:             1 
f4rank: 3 dof: 0 chisq: 0.000 tail:  i         1 
 dofdiff: 4 chisqdiff:  8.647 taildiff:  0.0705644793 
\end{verbatim}

\nld
Next we give the mixture coefficients and standard errors, which are typically far from independent.
Then an error covariance matrix, computed with the block jackknife.
\begin{verbatim}
best coefficients:     0.322     0.053     0.625
      std. errors:     0.177     0.114     0.099

error covariance (* 1000000)
     31255     -17297     -13957
    -17297      13044       4253
    -13957       4253       9704
\end{verbatim} 

\nld
We finally give an `all subsets analysis' where the coefficient under a '1' is forced
zero.
\begin{verbatim}
    fixed pat      dof     chisq       tail prob
          000 0      4     5.833        0.211948     0.322     0.053     0.625
          001 1      5    30.207               0     1.365    -0.365     0.000  infeasible
          010 1      5     6.101        0.296519     0.386     0.000     0.614
          100 1      5     9.903        0.078038     0.000     0.226     0.774
          011 2      6    37.560     1.36904e-06     1.000    -0.000     0.000
          101 2      6   158.115               0     0.000     1.000     0.000
          110 2      6    22.327      0.00105618     0.000    -0.000     1.000
\end{verbatim}
Here we see that, at least in this analysis there are reasonable models with CordedWareNeolithic
is a mix of either WHG or LBKNeolithic and YamnayaEBA.  
This is unsurprising, given the standard errors above.  
The point of this note is not to give a serious phylogenetic analysis but the results here certainly support 
a major Steppe contribution to the Corded Ware population, 
which is entirely concordant with the archaeology \cite{anthony}. 

\bibliography{../mybib}


\end{document}

